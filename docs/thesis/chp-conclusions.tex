%&tex

\chapter{Conclusion and discussion}\label{chp:conclusions}

In which we conclude our thesis, provide some discussion on the topics covered by pointing the
advantages and the flaws in using image classification in self-driving, and finally give a brief
overview on possible future work.

\section{Discussion}

In this thesis we argued the feasibility of lane following via image classification. We showed
that, with a fast and accurate model, it is possible to obtain reasonable results with such a
simple approach. SPNs were fast enough to be able to provide both accuracy and speed, even on such
a limited computer as the Raspberry Pi.

Having said that, we identified several flaws in this approach as documented by the previous
chapter. First and foremost, the choice of which model to use (i.e. to find the fine balance
between network complexity and inference speed) is still unquantifiable. At which point is a fast
model accurate enough? Should self-driving be more focused on being accurate or fast?

Furthermore, although current deep models have proven to be able to extract very meaningful
features and reach impressive accuracy levels, it is still not completely foolproof. Ambiguous
markings such as the ones mentioned in~\autoref{chp:realworld} can easily fool any model, as they
are perfectly valid classifications. These bring various dangers in both the physical and
metaphysical sense in terms of ethics.

Despite all this

\section{Further work}
