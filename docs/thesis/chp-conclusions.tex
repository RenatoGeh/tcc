%&tex

\chapter{Conclusion and discussion}\label{chp:conclusions}

In which we conclude our thesis, provide some discussion on the topics covered by pointing the
advantages and the flaws in using image classification in self-driving, and finally give a brief
overview on possible future work.

\section{Conclusion}

In this thesis we argued the feasibility of lane following via image classification. We showed
that, with a fast and accurate model, it is possible to obtain reasonable results with such a
simple approach. SPNs were fast enough to be able to provide both accuracy and speed, even on such
a limited miniature computer as the Raspberry Pi.

Having said that, we identified a few flaws in this approach as documented by the previous chapter.
First and foremost, the choice of which model to use (i.e. to find the fine balance between network
complexity and inference speed) is still unquantifiable. At which point is a fast model accurate
enough? Should self-driving be more focused on being accurate or fast?

Second, although current deep models have proven to be able to extract very meaningful features and
reach impressive accuracy levels, it is still not completely foolproof. Ambiguous markings such as
the ones mentioned in~\autoref{chp:realworld} can easily fool any model, as they are perfectly
valid classifications. A possible solution to this would be to account for previous classifications
through a temporal model, such as a markov chain like network or a recurrent neural network.

Finally, the choice of how to model the control system can play a big part in how well the robot is
going to perform on a real case scenario. We chose a very simple control system that only turned
left, right or went straight with no degree of continuity between commands. Furthermore, our
implementation had a fixed turn ratio, meaning sharp turns were a problem from the start.

Despite all this, our attempt was reasonably successful at modelling self-driving on a low-budget
mobile robot. The robot was able to correct itself before going off tracks, identified turns
correctly, and in its best iteration was able to follow long lines smoothly. It is also worthy to
mention that training was much faster and required fewer training samples than other deep models.
Because of linear exact inference in SPNs, we were able to accurately quantify uncertainty very
fast.

\section{Further work}

This section is dedicated to possible future work related to this thesis. We give brief suggestions
on how to improve the training and inference model as well as the robot's control system.

In our thesis, we only implemented the~\cite{gens-domingos} and~\cite{clustering} structure
learning algorithms. There have been many other architectures that have achieved better results
since. Future work on a comparison between these more recent state-of-the-art networks would be a
very interesting path to take.

For weight learning, we only applied hard generative and discriminative gradient descent. There
have been many advances on weight learning, including~\cite{baum-welch} through the use of Extended
Baum-Welch. Accuracy increased significantly though this new weight learning method, and we
speculate whether this would provide a boost to accuracy.

Nowadays, deep models make use of GPUs to accelerate both learning and inference. Our
implementation made use of only the CPU. GPU parallelization in SPNs is still in its early infancy,
but implementations such as~\cite{deep-learn-spn} have shown that a more connectionist approach
similar to deep neural networks could potentially increase its performance whilst maintaining its
probabilistic semantic.

Our robot control system was based on a very simplistic approach that does not mirror a real-world
implementation. A more complex controller could potentially enhance self-driving as a whole.
However an excedingly complex system could put too much load on the robot, decreasing its
performance.
