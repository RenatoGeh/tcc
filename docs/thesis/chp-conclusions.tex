%&tex

\chapter{Conclusion and discussion}\label{chp:conclusions}

In which we conclude our thesis, provide some discussion on the topics covered by pointing the
advantages and the flaws in using image classification in self-driving, and finally give a brief
overview on possible future work.

\section{Discussion}

In this thesis we argued the feasibility of lane following via image classification. We showed
that, with a fast and accurate model, it is possible to obtain reasonable results with such a
simple approach. SPNs were fast enough to be able to provide both accuracy and speed, even on such
a limited computer as the Raspberry Pi.

Having said that, we identified a few flaws in this approach as documented by the previous chapter.
First and foremost, the choice of which model to use (i.e. to find the fine balance between network
complexity and inference speed) is still unquantifiable. At which point is a fast model accurate
enough? Should self-driving be more focused on being accurate or fast?

Second, although current deep models have proven to be able to extract very meaningful features and
reach impressive accuracy levels, it is still not completely foolproof. Ambiguous markings such as
the ones mentioned in~\autoref{chp:realworld} can easily fool any model, as they are perfectly
valid classifications. A possible solution to this would be to account for previous classifications
through a temporal model, such as a markov chain or a recurrent neural network.

Finally, the choice of how to model the robot's control system can play a big part in how well the
robot is going to perform on a real case scenario. We chose a very simple control system that only
turned left, right or went straight with no degree of continuity between commands. Furthermore, our
implementation had a fixed turn ratio, meaning sharp turns were a problem from the start.

Despite all this, our attempt was reasonably successful at modelling self-driving on a low-budget
mobile robot. The robot was able to correct itself before going off tracks, identified turns
correctly, and in its best iteration was able to follow long lines smoothly.

\section{Further work}
