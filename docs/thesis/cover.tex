\thispagestyle{empty}
\begin{center}
  \vspace{-3.0cm}
  {\includegraphics[scale=0.2]{imgs/logo-usp.png}}\\
  {\textbf{\LARGE Mobile Robot Self-Driving Through Image Classification Using
        Discriminative Learning of Sum-Product Networks}}\\
  \vspace{2.0cm}
  \Large Undergraduate Thesis\\
  \vspace{2.5cm}
  \large\flushleft{Student: Renato Lui Geh\\
    Advisor: Prof.\ Denis Deratani Mauá}\\
  \vspace{2.5cm}
  \centering
  {\textbf{\uppercase{Institute of Mathematics and Statistics\\University of São Paulo}}}\\
  \vspace{1.5cm}
  {\Large\textbf{São Paulo, Brazil}}\\
  \vspace{0.25cm}
  {\Large\textbf{2018}}\\
\end{center}

\chapter*{Acknowledgements}

I would like to greatly thank my advisor, Prof.\ Denis Deratani Mauá, for the support and
attention, but most of all for the patience of having to read countless reports for both my
undergraduate research and undergraduate thesis, many of which were not short.

To my parents, Chen and Luiz, for making sure I wanted for nothing, for giving me the best
education possible, and for always assuring me of my abilities.

A special thank you to Maria Clara Cardoso, my best friend and partner in crime, whose support and
help were irreplaceble. Our conversations are always filled with laughter and I will always hold
them dear.

A warm thanks to my friends and colleagues Ricardo Fonseca and Yan Couto, whose camaraderie and
friendship I cherish deeply.

I'd also like to express great gratitude to all professors I had during my undergraduate, whose
trade is often overlooked and underappreciated, yet managed to teach us so much and inspire us to
do our best.

\vfill
\makebox[\textwidth][r]{%
  \begin{minipage}{0.45\textwidth}
    This work was partially supported by CNPq grant PIBIC 800585/2016.\\
    \rule{\textwidth}{1.0pt}
  \end{minipage}
}

\chapter*{Abstract}

\noindent GEH, L. R. \textbf{Mobile robot self-driving through image classification using
  discriminative learning of sum-product networks}. Institute of Mathematics and Statistics,
University of São Paulo, São Paulo, Brazil, 2018.\\

Driving has proven to be a very difficult task for machines to emulate, not only due to the
inherent complexity of the problem but also because of the need for accurate real-time predictions.
Nonetheless, recent advances in computer vision and machine learning have shown promising results
in real-world applications. Mobile robots are low-cost small robots with limited processing power
and memory. The problem of self-driving can be similarly applied to the mobile robot domain as a
down-scaled version of the same task, with an additional hardware constraint. Sum-product networks
are probabilistic graphical models capable of representing tractable probability distributions
containing a great number of variables.  Exact inference is asymptotically linear to the number of
edges in the network's graph, and its deep architecture is capable of representing a wide range of
distributions. In this work, we attempt to model autonomous driving by using sum-product networks
on a small mobile robot. We model this task as an imitation learning problem through image
classification. We present accuracy results on an artificial self-driving dataset for different
sum-product network learning algorithms, providing a comparative study not only for different
network architectures, but also discriminative and generative models. Finally, we provide a
real-world mobile robot implementation on a miniature computer.\\

\noindent\textbf{Keywords:} sum-product networks, machine learning, robotics
