\usepackage{etoolbox}

\usepackage{ifxetex}
\usepackage{ifluatex}
\usepackage{ifpdf}
\usepackage[T1]{fontenc}
\usepackage[utf8]{inputenc}
\usepackage{amsthm}
\usepackage{babel}
\usepackage{enumitem}
\babeltags{br = brazil, en = english}

\newcommand\microtypeopts{
  protrusion=true,
  tracking=false,
  kerning=false,
  spacing=false
}

\ifxetex
  \usepackage[expansion=false,\microtypeopts]{microtype}
\else
  \usepackage[expansion=true,\microtypeopts]{microtype}
\fi

\tolerance=800
\hyphenpenalty=800
\setlength{\emergencystretch}{2.5em}

\frenchspacing

\usepackage[bottom]{footmisc}
\usepackage{emptypage}
\usepackage{setspace}

\usepackage{indentfirst}
\newlength\oldparindent
\setlength\oldparindent\parindent
\usepackage[parfill]{parskip}
\setlength\parindent\oldparindent

\usepackage[usenames,svgnames,dvipsnames]{xcolor}
\usepackage[newcommands]{ragged2e}

\makeatletter
\g@addto@macro{\centering}{\hbadness=\@M}
\g@addto@macro{\Centering}{\hbadness=\@M}
\g@addto@macro{\raggedright}{\hbadness=\@M}
\g@addto@macro{\RaggedRight}{\hbadness=\@M}
\g@addto@macro{\raggedleft}{\hbadness=\@M}
\g@addto@macro{\RaggedLeft}{\hbadness=\@M}
\g@addto@macro{\center}{\hbadness=\@M}
\g@addto@macro{\Center}{\hbadness=\@M}
\g@addto@macro{\flushleft}{\hbadness=\@M}
\g@addto@macro{\FlushLeft}{\hbadness=\@M}
\g@addto@macro{\flushright}{\hbadness=\@M}
\g@addto@macro{\FlushRight}{\hbadness=\@M}
\makeatother

\usepackage{textcase}
\usepackage{lmodern}
\usepackage{fix-cm}

\fontencoding{TS1}\fontfamily{lmr}\selectfont
\DeclareFontShape{TS1}{lmr}{b}{sc}{<->ssub * cmr/bx/n}{}
\DeclareFontShape{TS1}{lmr}{bx}{sc}{<->ssub * cmr/bx/n}{}

\fontencoding{T1}\fontfamily{lmr}\selectfont
\DeclareFontShape{T1}{lmr}{b}{sc}{<->ssub * cmr/bx/sc}{}
\DeclareFontShape{T1}{lmr}{bx}{sc}{<->ssub * cmr/bx/sc}{}

\fontencoding{T1}\fontfamily{lmr}\selectfont % já feito acima, mas tudo bem
\DeclareFontShape{T1}{lmr}{m}{scit}{<->ssub * lmr/m/scsl}{}
\DeclareFontShape{T1}{lmr}{bx}{scit}{<->ssub * lmr/bx/scsl}{}

\usepackage{inconsolata}
\usepackage[mono=false,nofontspec]{libertine}
\usepackage[libertine]{newtxmath}
\usepackage{slantsc}
\usepackage{fancyhdr}
\renewcommand{\headrulewidth}{0pt}

\setcounter{tocdepth}{2}

\dimdef{\othermargin}{\paperwidth-1in-\hoffset-\oddsidemargin-\textwidth}
\addtolength{\headwidth}{\othermargin}
\addtolength{\headwidth}{-10mm}

\newcommand{\formataNumPaginas}{
  \fancyhead[RO]{\raisebox{8mm}{\footnotesize\bfseries\thepage}}
  \fancyhead[LE]{\raisebox{8mm}{\footnotesize\bfseries\thepage}}
}

\newcommand{\formataCabecalhosDinamicos}{
  \fancyhead[LO]{\scriptsize\MakeTextUppercase{\rightmark}}
  \fancyhead[RE]{\scriptsize\MakeTextUppercase{\leftmark}}
}

\fancypagestyle{mainmatter}{
  \renewcommand{\chaptermark}[1]{
    \markboth
      {\thechapter\hskip 0.3em |\hskip 0.3em ##1}
      {\thechapter\hskip 0.3em |\hskip 0.3em ##1}
  }

  \renewcommand{\sectionmark}[1]{
    \markright{\thesection\hskip 0.3em |\hskip 0.3em ##1}
  }

  \fancyhf{}
  \formataNumPaginas{}
  \formataCabecalhosDinamicos{}
}

\fancypagestyle{appendix}{
  \renewcommand{\chaptermark}[1]{%
    \markboth{%
      \appendixname\ \thechapter%
    }{%
      \thechapter\hskip 0.3em |\hskip 0.3em ##1%
    }
  }

  \fancyhf{}
  \formataNumPaginas{}
  \formataCabecalhosDinamicos{}
}

\fancypagestyle{frontback}{
  \renewcommand{\chaptermark}[1]{\markboth{##1}{##1}}
  \renewcommand{\sectionmark}[1]{\markboth{##1}{##1}}

  \fancyhf{}
  \formataNumPaginas{}
  \formataCabecalhosDinamicos{}
}

\fancypagestyle{plain}{
  \fancyhf{}
  \formataNumPaginas{}
}

\usepackage{graphicx}

\usepackage{float}
\floatplacement{table}{htbp}
\floatplacement{figure}{htbp}
\usepackage{flafter}
\usepackage[section,above,below]{placeins}

\renewcommand{\topfraction}{.85}
\renewcommand{\dbltopfraction}{.66}
\renewcommand{\bottomfraction}{.7}
\renewcommand{\textfraction}{.15}
\renewcommand{\floatpagefraction}{.66}
\renewcommand{\dblfloatpagefraction}{.66}
\setcounter{topnumber}{9}
\setcounter{dbltopnumber}{9}
\setcounter{bottomnumber}{9}
\setcounter{totalnumber}{20}

\usepackage{pdflscape}
\usepackage[figuresright,clockwise]{rotating}

\usepackage[
  font=small,
  format=plain,
  labelfont=bf,up,
  textfont=it,up]{caption}

\usepackage{subcaption}
\usepackage{wrapfig}
\usepackage{pdfpages}


\usepackage{tablefootnote}
\usepackage{array}

\newcolumntype{M}[1]{>{\centering}m{#1\textwidth}}
\newcolumntype{L}[1]{>{\RaggedRight}m{#1\textwidth}}
\newcolumntype{R}[1]{>{\RaggedLeft}m{#1\textwidth}}
\newcolumntype{J}[1]{m{#1\textwidth}}

\usepackage{dcolumn}
\usepackage{longtable}
\usepackage{multirow}
\usepackage{makecell}
\renewcommand{\rothead}[2][60]{\makebox[11mm][l]{\rotatebox{#1}{\makecell[c]{#2}}}}

\newlength\savedwidth
\newcommand\thickhline{
  \noalign{
    \global\savedwidth\arrayrulewidth
    \global\arrayrulewidth 1.5pt
  }
  \hline
  \noalign{\global\arrayrulewidth\savedwidth}
}

\usepackage{booktabs}

\usepackage[nottoc,notlot,notlof]{tocbibind}

\begin{filecontents*}{hyperxindy.xdy}
(define-attributes ("emph"))
(markup-locref :open "\hyperpage{" :close "}" :attr "default")
(markup-locref :open "\textbf{\hyperpage{" :close "}}" :attr "textbf")
(markup-locref :open "\textit{\hyperpage{" :close "}}" :attr "textit")
(markup-locref :open "\emph{\hyperpage{" :close "}}" :attr "emph")
\end{filecontents*}

\begin{filecontents*}{mkidxhead.ist}
headings_flag 1
heading_prefix "{\\bfseries "
heading_suffix "}\\nopagebreak\n"
\end{filecontents*}

\usepackage{imakeidx}

\indexsetup{
  othercode={\chaptermark{\indexname}},
}

\makeindex[
  noautomatic,
  intoc,
  options=-s mkidxhead.ist -l -L,
]

\usepackage{verbatim}

\usepackage{latexsym}
\usepackage{amsmath}
\usepackage{amssymb}
\usepackage{mathtools}

\usepackage{csquotes}

\usepackage[
  unicode=true,
  plainpages=false,
  pdfpagelabels,
  colorlinks=true,
  %citecolor=black,
  %linkcolor=black,
  %urlcolor=black,
  %filecolor=black,
  citecolor=DarkGreen,
  linkcolor=NavyBlue,
  urlcolor=DarkRed,
  filecolor=green,
  bookmarksopen=true,
]{hyperref}

\usepackage[all]{hypcap}
\usepackage{hyperxmp}

\usepackage{hyperref}
\urlstyle{sf}

\usepackage{listings}

\lstset{
  basicstyle=\footnotesize\ttfamily,  % the font that is used for the code
  numbers=left,                       % where to put the line-numbers
  numberstyle=\footnotesize\ttfamily, % the font that is used for the line-numbers
  stepnumber=1,                       % the step between two line-numbers. If it's 1 each line will be numbered
  numbersep=20pt,                     % how far the line-numbers are from the code
  commentstyle=\color{Brown}\upshape,
  stringstyle=\color{black},
  identifierstyle=\color{DarkBlue},
  keywordstyle=\color{cyan},
  showspaces=false,                   % show spaces adding particular underscores
  showstringspaces=false,             % underline spaces within strings
  showtabs=false,                     % show tabs within strings adding particular underscores
  frame=single,                       % adds a frame around the code
  framerule=0.6pt,
  tabsize=2,                          % sets default tabsize to 2 spaces
  captionpos=b,                       % sets the caption-position to bottom
  breaklines=true,                    % sets automatic line breaking
  breakatwhitespace=false,            % sets if automatic breaks should only happen at whitespace
  escapeinside={\%*}{*},             % if you want to add a comment within your code
  backgroundcolor=\color[rgb]{1.0,1.0,1.0}, % choose the background color.
  rulecolor=\color{darkgray},
  extendedchars=true,
  inputencoding=utf8,
  xleftmargin=30pt,
  xrightmargin=10pt,
  framexleftmargin=25pt,
  framexrightmargin=5pt,
  framesep=5pt,
}

\lstdefinestyle{wider} {
  tabsize = 4,
  numbersep=15pt,
  xleftmargin=25pt,
  framexleftmargin=20pt,
}

\lstdefinestyle{nocolor} {
  commentstyle=\color{darkgray}\upshape,
  stringstyle=\color{black},
  identifierstyle=\color{black},
  keywordstyle=\color{black},
}

\lstdefinelanguage{XML}{
  morecomment=[s]{<!--}{-->},
  morecomment=[s]{<!-- }{ -->},
  morecomment=[n]{<!--}{-->},
  morecomment=[n]{<!-- }{ -->},
  morestring=[b],
  morestring=[s]{>}{<},
  morecomment=[s]{<?}{?>},
  morekeywords={xmlns,version,type}% list your attributes here
}

\usepackage{algorithm}
\usepackage[noend]{algpseudocode}

\usepackage{tikz}
\usetikzlibrary{positioning}
\usetikzlibrary{shapes.geometric}
\usetikzlibrary{fit}
\usetikzlibrary{calc}

\usepackage{textcomp}
\usepackage{multicol}

\usepackage{cleveref}

\usepackage{epigraph}
\newcommand{\epigrafe}[2] {%
  \setlength{\epigraphrule}{0pt}
  \ifthenelse{\equal{}{#2}}{
    \epigraph{\itshape\RaggedLeft #1}{}
  }{
    \epigraph{\itshape\RaggedLeft #1}{--- #2}
  }
}

\newrobustcmd{\NoChangeOrSentenceCase}[1]{%
  \ifthenelse{\ifcurrentfield{booktitle}\OR\ifcurrentfield{booksubtitle}%
    \OR\ifcurrentfield{maintitle}\OR\ifcurrentfield{mainsubtitle}%
    \OR\ifcurrentfield{journaltitle}\OR\ifcurrentfield{journalsubtitle}%
    \OR\ifcurrentfield{issuetitle}\OR\ifcurrentfield{issuesubtitle}%
    \OR\ifentrytype{book}\OR\ifentrytype{mvbook}\OR\ifentrytype{bookinbook}%
    \OR\ifentrytype{booklet}\OR\ifentrytype{suppbook}%
    \OR\ifentrytype{collection}\OR\ifentrytype{mvcollection}%
    \OR\ifentrytype{suppcollection}\OR\ifentrytype{manual}%
    \OR\ifentrytype{periodical}\OR\ifentrytype{suppperiodical}%
    \OR\ifentrytype{proceedings}\OR\ifentrytype{mvproceedings}%
    \OR\ifentrytype{reference}\OR\ifentrytype{mvreference}%
    \OR\ifentrytype{report}\OR\ifentrytype{thesis}%
    \OR\ifentrytype{online}\OR\ifentrytype{misc}}
    {#1}
    {\MakeSentenceCase*{#1}}}
